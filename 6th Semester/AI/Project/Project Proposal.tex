\documentclass{article}
\usepackage[utf8]{inputenc}
\usepackage{natbib}
\usepackage{graphicx}

\title{Artificial Intelligence Project Proposal}
\author{
    Shakleen Ishfar\\
    ID: 160041029
    \and
    Abir Ahbab\\
    ID: 160041032
    }
\date{11 July, 2019}

\begin{document}

\maketitle

\section{Title}
The title of the project is \lq\lq\textbf{Effective Transportation System}\rq\rq.

\section{Description}
\subsection{Aims and Objectives}
The aim of this project is to solve the problems that plague transportation facilities. For any institution transportation facilities are a must. But most institutions can't setup the facilities effectively. This ends up costing the institution a fortune and end up being a hassle for the passengers. 
\newline
The aim of this project is to build an effective transportation system that will have the following outcomes:
\begin{enumerate}
    \item Minimize the initial cost of creating the system.
    \item Minimize passenger fee for using the system.
    \item Maximize the profit of the system.
\end{enumerate}

\subsection{Rationale and Subject Matter}
Transportation facilities are a key part of any institution in this modern age. However, building them effectively are easier said than done. There are many constraints that make the process difficult, such as the number of vehicles to use, the number of routes to take into consideration etc. I study in Islamic University of Technology, in the department of Computer Science and Engineering. The university is located at Boardbazar, Gazipur. There are a large number of students and faculty members who will be largely benefitted by such a system. The motivation of this project stems from the necessity of such facilities.

\subsection{Expected Results}
The aim of this project is to create a system model to produce the best possible way to build the transportation facilities for a corporation. The system will take the constraints of the transportation system some of which are as follows:
\begin{enumerate}
    \item Vehicle price
    \item Fuel cost
    \item Number of people using transportation facilities
\end{enumerate}
The specific properties that the facilities should have will be outputted by the system. For example:
\begin{enumerate}
    \item Profit of the system
    \item Number of vehicles
    \item Number of routes
    \item Transportation cost for each passenger based on distance
\end{enumerate}

\subsection{Relevance to National Development}
Bangladesh is a country with a huge population and limited resources. A staggering amount of people spend everyday travelling from one place to another for the sake of their livelihood. However, the transportation system of the country is not up-to the task of handling this effectively. As such, the transportation system ends up wasting a lot of resources and causes people to endure many problems and difficulties. It is the hoped that this project will be able to tackle many of these problems that have plagued our transportation system and will be able to offer effective solutions to develop the existing facilities.

\subsection{Methodology of Investigation}
To build an effective system some background research and investigation is required. For this the following methodology will be followed:
\begin{itemize}
    \item Investigate the needs of the system from the side of users and investors.
    \item Research about transportation systems. Find the advantages and disadvantages of each system.
\end{itemize}

\end{document}