\documentclass[../main.tex]{subfiles}
 
\begin{document}
\section{First Task}
\subsection{Question: Task 1}
\emph{Write MATLAB code that will take a sinusoid and quantize it using b bits. Use the sinusoid x(t) = sin(t) in the interval t=0:0.1:4*pi. Quantize x(t) using b = 4, 8, 12 and 16 bits. For each case of b, plot the original signal, quantized signal and the quantization error.}

\subsection{Theory: Quantization}
An analog signal has no discrete number of values. This type of signal can take on any value within a specific range. However, this signal can't be processed efficiently due to this reason. Which is why for ease of storing and processing an analog signal is converted into a digital signal. The process of converting from analog to digital has 3 steps which are as follows:
\begin{enumerate}
    \item \textbf{Sampling:} The process of taking finite sets of readings. Turns continuous signals to discrete signals.
    \item \textbf{Quantization:} The process of taking only discrete values per reading. Turns analog signals to digital signals.
    \item \textbf{Encoding:} Giving each quantized value a specific code for representation in digital format.
\end{enumerate}

The steps of \emph{quantization} are listed below:
\begin{enumerate}
    \item Take the difference of min and max amplitude and save it as \emph{range}. 
    \[range = max(amplitude) - min(amplitude)\]
    
    \item To perform quantization we need to know how many discrete value to divide \emph{range} into. Usually, this is a power of 2. So if we denote this as \emph{levels} we can write
    \[levels = 2^n\]
    \emph{where n is the number of encoding bits we wish to use.}
    
    \item Next, we find the \emph{step value} for quantization.
    \[step value = \frac{range}{levels}\]
    
    \item This step value is the smallest step size in the quantization process. An analog reading will be converted to the closest discrete value which is a multiple of this step value in either the positive or negative direction.
\end{enumerate}

\subsection{MATLAB Code: quantize()}
The code listed below demonstrates the quantization process.
\lstinputlisting[language=Octave]{./Codes/quantize.m}

\subsection{Test Run: quantize()}
Running the function \emph{quantize(signal, bits)} for sin wave as signal and different values of bits
\lstinputlisting[language=Octave]{./Codes/quantize_test.m}
give the plots in \emph{Figure 1}. In each plot, the analog and digital signals are shown simultaneously.

\begin{figure}[ht]
    \caption{Quantization of analog signals}
    \includegraphics[width=6cm, height=5cm]{Quantize_4.jpg}
    \includegraphics[width=6cm, height=5cm]{Quantize_8.jpg}
    \includegraphics[width=6cm, height=5cm]{Quantize_12.jpg}
    \includegraphics[width=6cm, height=5cm]{Quantize_16.jpg}
\end{figure}
\end{document}

