\documentclass[../main.tex]{subfiles}
 
\begin{document}
\section{Fourth Task}
\subsection{Question: Task 4}
\emph{Write a MATLAB function sigshift that takes a signal x(n) and a shift value k as inputs and returns the resulting signal y(n) = x(n-k).}

\subsection{Theory: Shifting Signals}
Shifting a signal means getting the values earlier or later. Mathematically shifting is represented as $y(n) = x(n \pm k)$ Shifting is of two types which are:
\begin{enumerate}
    \item Delayed: Shifting a signal by a negative amount. k is negative.
    \item Early: Shifting a signal by a positive amount. k is positive.
\end{enumerate}

\subsection{MATLAB Code: sigshift()}
The code listed below implements a function that shifts a signal by k units.
\lstinputlisting[language=Octave]{./Codes/sigshift.m}

\subsection{Test Run: sigshift()}
Running the function sigshift(inputSignal, k) for the signal $1, 2, 3, 4, 5$ and k = 2
\lstinputlisting[language=Octave]{./Codes/sigshift_test.m}
gives the result $0, 0, 1, 2, 3$ and shows the plot in \emph{Figure 6}.

\begin{figure}[ht]
    \centering
    \caption{Shifting signals}
    \includegraphics[width=6cm, height=5cm]{Sigshift.jpg}
\end{figure}
\end{document}

