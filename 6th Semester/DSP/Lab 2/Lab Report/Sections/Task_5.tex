\documentclass[../main.tex]{subfiles}
 
\begin{document}

\section{Fifth Task}
\subsection{Question: Task 5}
\emph{Write a MATLAB function sigfold that takes a signal x(n) and returns the resulting signal y(n) = x(-n). Verify the correctness of your function by taking suitable signals as input/output and plotting them.}

\subsection{Theory: Signal Folding}
Folding a signal means reversing the axis the signal is on. Basically, shifting means $y(n) = x(-n)$. When a signal is folded the origin remains unchanged. However, the values on the positive part of the axis goes to the negative part of the axis and vise versa.

\subsection{MATLAB Code: sigfold()}
The code listed below implements a function that folds a signal.
\lstinputlisting[language=Octave]{./Codes/sigfold.m}

\subsection{Test Run: sigfold()}
Running the function sigfold(inputSignal) for the signal $1, 4, 9, ..., 100$
\lstinputlisting[language=Octave]{./Codes/sigfold_test.m}
gives the output $100, 81, 64, ..., 9, 4, 1$ and shows the plot in \emph{Figure 7}.

\begin{figure}[ht]
    \centering
    \caption{Folding signal}
    \includegraphics[width=6cm, height=5cm]{Sigfold.jpg}
\end{figure}

\end{document}