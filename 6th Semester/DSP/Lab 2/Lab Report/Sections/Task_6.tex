\documentclass[../main.tex]{subfiles}
 
\begin{document}

\section{Sixth Task}
\subsection{Question: Task 6}
\emph{Write a MATLAB function downsample that takes a signal x(n) and a value d (d is an integer, d>1) and returns the down-sampled signal y(n) = x(dn). Verify the correctness by plotting suitable input/output signals.}

\subsection{Theory: Down Sampling}
Down sampling a signal means taking a smaller subset of readings of the signal. Mathematically, $y(n) = x(dn)$ where d is the amount of down sampling applied.

\subsection{MATLAB Code: downsample()}
The code listed below implements a function that downsamples a signal.
\lstinputlisting[language=Octave]{./Codes/downsample.m}

\subsection{Test Run: downsample()}
Running the function downsample(inputSignal, d) for the signal $1, 4, 9, ..., 144$ and d = 2, 3
\lstinputlisting[language=Octave]{./Codes/downsample_test.m}
gives the result $0, 4, 0, 16, 0, 36, 0, 64, 0, 100, 0, 144$ and shows the plot in \emph{Figure 8}.

\begin{figure}[ht]
    \centering
    \caption{Down Sampling Signals}
    \includegraphics[width=6cm, height=5cm]{./Figures/Down_Sample_2.jpg}
    \includegraphics[width=6cm, height=5cm]{./Figures/Down_Sample_3.jpg}
\end{figure}

\end{document}