\documentclass{article}
\usepackage[utf8]{inputenc}
\usepackage{listings}
\usepackage{graphicx}

\title{Digital Signal Processing - Lab 2 Report}
\author{Shakleen Ishfar}
\date{08 July, 2019}


\begin{document}

\maketitle


\section{First Task}
\subsection{Question}
\emph{Write MATLAB code that will take a sinusoid and quantize it using b bits. Use the sinusoid x(t) =sin(t) in the interval t=0:0.1:4*pi. Quantize x(t) using b = 4, 8, 12 and 16 bits. For each case of b, plot the original signal, quantized signal and the quantization error.}

\subsection{Theory}

\subsection{MATLAB Code}
\lstinputlisting[language=Octave]{./Codes/quantize.m}

\subsection{Test Run}
Running these commands 
\begin{verbatim} 
quantize(@(x)sin(x), 4);
quantize(@(x)sin(x), 8);
quantize(@(x)sin(x), 12);
quantize(@(x)sin(x), 16);
\end{verbatim}
give the following plots. In each plot, the analog and digital signals are shown simultaneously.

\begin{figure}[h]
    \caption{Quantization of analog signals}
    \includegraphics[width=6cm, height=5cm]{./Figures/Quantize_4.jpg}
    \includegraphics[width=6cm, height=5cm]{./Figures/Quantize_8.jpg}
    \includegraphics[width=6cm, height=5cm]{./Figures/Quantize_12.jpg}
    \includegraphics[width=6cm, height=5cm]{./Figures/Quantize_16.jpg}
\end{figure}

\section{Second Task}
\subsection{Question}
\emph{Write MATLAB functions delta(n), unity(n) and unitramp(n) which will depict the elementary signals we read about in the class. Each of this functions, for a given value of n (n>0), plots the corresponding signals in the range of –n to n.}

\subsection{Theory of Delta Signal}

\subsection{MATLAB Code of Delta Signal}
\lstinputlisting[language=Octave]{./Codes/delta.m}

\subsection{Test Run of Delta Signal}
Running this command 
\begin{verbatim} 
delta(10);
\end{verbatim}
gives the following plot:

\begin{figure}[h]
    \centering
    \caption{Delta elementary signal}
    \includegraphics[width=6cm, height=5cm]{./Figures/Delta.jpg}
\end{figure}

\subsection{Theory of Unity Signal}

\subsection{MATLAB Code of Unity Signal}
\lstinputlisting[language=Octave]{./Codes/unity.m}

\subsection{Test Run of Unity Signal}
Running this command 
\begin{verbatim} 
unity(10);
\end{verbatim}
gives the following plot:

\begin{figure}[h]
    \centering
    \caption{Unity elementary signal}
    \includegraphics[width=6cm, height=5cm]{./Figures/Unity.jpg}
\end{figure}

\subsection{Theory of Unitramp Signal}

\subsection{MATLAB Code of Unitramp Signal}
\lstinputlisting[language=Octave]{./Codes/uniramp.m}

\subsection{Test Run of Unitramp Signal}
Running this command 
\begin{verbatim} 
unitramp(10);
\end{verbatim}
gives the following plot:

\begin{figure}[h]
    \centering
    \caption{Delta elementary signal}
    \includegraphics[width=6cm, height=5cm]{./Figures/Unitramp.jpg}
\end{figure}

\section{Third Task}
\subsection{Question}
\emph{Write a MATLAB function that will take as input an arbitrary signal x(n) and divide it into Symmetric (even) and Antisymmetric (odd) parts and plots the three signals (original signal, even part and odd part) in the same plot.}

\subsection{Theory}

\subsection{MATLAB Code}
\lstinputlisting[language=Octave]{./Codes/evenOddFunction.m}

\bibliographystyle{plain}
\bibliography{references}
\end{document}
