\documentclass{article}
\usepackage[utf8]{inputenc}

\title{Technical Report Writing Lab}
\author{Shakleen Ishfar}
\date{08 July, 2019}

\usepackage{natbib}
\usepackage{graphicx}

\begin{document}

\maketitle

\section{Research Domain}
I've selected \emph{Network and Data Analysis} as my research domain.

\section{Reasons}
We live in the age of information. Information is data presented in a meaningful way. From the people around us to the machines we use regularly, everything is producing data constantly. To harness the power of data, it must be converted to useful information. To do this, we need to collect, process and analyze data effectively.
\newline
The domain of network and data analysis deals with the collection, processing and analysis of data. Internet is a huge source of data. The effective use of this data can lead to many ground breaking discoveries and inventions, which could in turn, revolutionize our civilization. The promising research happening centering data and networking excites me. Which is why I picked \emph{Network and Data Analysis} as my research domain.

\section{Motivation}
Data is everything in this day and age. By properly analyzing datasets, one can discover the pattern hidden within them. This pattern can be used to accomplish great ordeals. However, to do so, one needs to properly collect, process and analyze this data. Only then, can one make sense of it.
\newline
Large companies like \emph{Google}, \emph{Microsoft}, \emph{Amazon}, \emph{Facebook} etc. has put billions into the research centering data analysis. These companies are betting that our future will undeniably dependent on data and that the progress of mankind will lie in the proper utilization of it. Huge progress has been made in the field of speech recognition, image and signal processing etc. by using data. Facebook has used the data of their huge user base to train their face recognition algorithm, which can accurately recognize a person from a photo. Google makes majority of their revenue from their \emph{Adware} program. Google has used data from users using their services to build an effective ad and marketing program. This program targets the ad of a product to audiences who are more likely to buy that product. MIT has recently been successful in predicting and pinpointing breast cancer as early as 5 years by using state of the art machine learning models and training the models on huge amounts of data. This is how data is revolutionizing our everyday life and making things better for us.
\newline
The promise in data motivates me. Bangladesh has a huge amount of people. These people are constantly producing data and this huge amount of data is left untapped. It is my belief, by tapping into this huge source of data, we can find solutions to many of the problems plaguing the country. The way to move forward to a \emph{Digital Bangladesh} is unimaginable without the effective utilization of data. 

\section{List of papers}

\begin{enumerate}
    \item \textbf{Opportunities for multi-agent systems and multi-agent reinforcement learning in traffic control}\newline
    \textbf{Author(s):} Ana L. C. BazzanEmail\newline
    \textbf{Published in:} 07 September, 2008 in Springer Link
    
    \item \textbf{A two-stage road traffic congestion prediction and resource dispatching toward a self-organizing traffic control system}\newline
    \textbf{Author(s):} Zied Bouyahia, Hedi Haddad, Nafaa Jabeur, Ansar Yasar\newline
    \textbf{Published in:} 27 March, 2019 in Springer Link
    
    \item \textbf{The simulation model on delay time of road accessibility based on intelligent traffic control system}\newline
    \textbf{Author(s):}Zhichao Li, Ru Jia, Jilin Huang\newline
    \textbf{Published in:} 04 January, 2018 in Springer Link
    
    \item \textbf{Cell phone big data to compute mobility scenarios for future smart cities}\newline
    \textbf{Author(s):} Davide Tosi\newline
    \textbf{Published in:} 28 June, 2017 in Springer Link
    
    \item \textbf{Internet of Things Meets Brain–Computer Interface: A Unified Deep Learning Framework for Enabling Human-Thing Cognitive Interactivity}\newline
    \textbf{Author(s):} Xiang Zhang, Lina Yao, Shuai Zhang, Salil Kanhere, Michael Sheng, Yunhao Liu\newline
    \textbf{Published in:} 24 October, 2018 in IEEE Internet of Things Journal
\end{enumerate}

\bibliographystyle{plain}
\bibliography{references}
\end{document}
