\documentclass{article}
\usepackage[utf8]{inputenc}

\title{Technical Report Writing Assignment 2}
\author{
    Shakleen Ishfar\\
    ID: 160041029
    }
\date{22 July, 2019}

\begin{document}

\maketitle

\section{First Paragraph}
Deliberate act of foolishness causes pollution which is a threat to our existence and it needs to be minimized through proper planning.
\newline
Pollution, a major threat to our survival, can be considered as direct or indirect change in any component of the biosphere. It's caused by sudden and rapid industrialization. The toxic materials and pollutants threaten living beings with serious health risks. Air polluted from smoke and dust harm our lungs. Toxic drinking water is spreading dangerous diseases. Sound pollution from construction causes stress, sleeping disorder and also damages hearing capabilities. We are reluctant to change our deliberate acts of polluting the environment being fully aware that such activities are destroying the earth. To prevent pollution, the governemnt has envoked many policies like adoption of clean and low waste technologies, reusing and recycling etc. Moreover, “Pollution prevention approach”  seeks to increase the efficiency of a process reducing the amount of pollution generated at its source. Various steps initiated to give effect to the government policies include statutory stringent regulations, development of environmental standards, control of pollution generated through vehicles etc.

\section{Second Paragraph}
The 7th to the 13th century was the golden age of Muslim learning. Muslim scholars made some great contributions to mathematics, astronomy, geography etc. Some famous muslim scholars are:

\textbf{AL-KHWARIZMI (780 - 850 CE)} is known as the father of algebra. He was born in Kath. Algebra derived it's name from his book Hisab Al-Jabr wal Mugabalah. In Cambridge University, a latin translation of 'Algoritimi de Numero Indorum', Al-Khowarizmi's arithmetic text, was found. In the twelfth century, English scholars translated his books into Latin. Mathematicians all over the world, used these books until the sixteenth century.

\textbf{AL-KINDI (801-873 CE)}, was born in Kufa during the governorship of his father. He was known as Faylasuf Al-Arab to his people. He wrote eleven texts on numbers and numerical analysis.

\textbf{AL-KARAJI} was born in Kharkh, Baghdad. His book ‘Al-Kafi fi Al-Hisab' covers the rules of computation. His second book, ‘Al- Fakhri' named after his friend, the Grand Vizier of Baghdad.

\textbf{Al-BATTANI (850-929 CE)} known as the father of trigonometry and is considered to be the greatest Muslim astronomer and mathematician. He computed the first table of cotangents.

\textbf{AL-BIRUNI (973-1050 CE)} discussed the theory of the earth rotating about its own axis, determined the earth's circumference, enabled the direction of the Qibla to be determined from anywhere in the world.

The theory of the functions was developed by Muslim scholars of the tenth century. They worked diligently in the development of plane and spherical trigonometry. It is based on Ptolemy's theorem but is superior. Because it employs the sine where Ptolemy used the chord and is in algebraic instead of geometric form.


\end{document}
