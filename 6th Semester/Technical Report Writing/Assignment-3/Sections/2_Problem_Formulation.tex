\documentclass[../main.tex]{subfiles}

\begin{document}

\section{Problem Formulation}
\subsection{Spanning tree}
A \emph{spanning tree} of a connected and undirected graph is a subgraph that is a tree connecting all the vertices together. A single graph can have many different spanning trees. The weight of a spanning tree is the sum of weights given to each edge of the spanning tree.

\subsection{Minimum Spanning Tree}
A \emph{minimum spanning tree (MST)}, for a weighted, connected and undirected graph, is a spanning tree with weight less than or equal to the weight of every other spanning tree. Let a graph have V vertices. Then the MST of that graph will have exactly V-1 number of edges.
asd
\end{document}